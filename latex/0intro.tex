\chapter*{Introduction}
\addcontentsline{toc}{chapter}{Introduction}
\label{introduction}
Ce rapport est le rapport du projet de \textit{ray tracing} donné dans le cadre du cours d'ELECH304 - Physique des Télécommunications, cours donné en troisième année de bachelier en cursus d'ingénieur civil en option physique et en option électronique et télécommunication.\\
L'objectif du projet est de réaliser un code de calcul qui permet d'analyser la couverture réseau d'antennes Wifi 802.11.
Celle-ci est simulée via ray tracing en utilisant la méthode des images. Plusieurs simplifications physiques et géométriques sont prises en compte dans la réalisation, et elles sont détaillées dans la section \TODO{appropriée}.\\
L'espace dans lequel la simulation est faite est un appartement composé de plusieurs pièces, avec des murs de matériau différents et donc de propriétés physiques différentes. \\
L'objectif final est d'optimiser la couverture de réseau dans l'appartement entier, en plaçant intelligemment les émetteurs via des méthodes d'optimisation que l'on \TODO{verra plus loin}.
