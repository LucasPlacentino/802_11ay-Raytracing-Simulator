\chapter*{Introduction}
\addcontentsline{toc}{chapter}{Introduction}
\label{introduction}
Ce rapport présente le projet de \textit{ray tracing} donné dans le cadre du cours d'ELECH304 - Physique des Télécommunications, cours donné en troisième année de bachelier en cursus d'ingénieur civil en option physique et en option électronique et télécommunication.
L'objectif du projet est de réaliser un code de calcul qui permet d'analyser la couverture réseau d'antennes Wi-Fi IEEE 802.11ay.
Celle-ci est simulée via ray-tracing en utilisant la méthode des images. Plusieurs simplifications physiques et géométriques sont détaillées puis prises en compte dans la réalisation.
L'espace dans lequel la simulation est faite est un appartement composé de plusieurs pièces, avec des murs de matériau différents et donc de propriétés physiques différentes.

Une application dotée d'une interface graphique a été réalisée afin de simplifier le lancement et la paramétrisation de la simulation.

L'objectif final est d'observer la couverture du réseau dans l'appartement entier, afin d'ensuite optimiser le placement de la station de base pour maximiser cette couverture à une vitesse la plus haute possible.% via des méthodes d'optimisation que l'on \TODO{verra plus loin}, tel que la méthode Monte-Carlo ou encore un algorithme génétique.
