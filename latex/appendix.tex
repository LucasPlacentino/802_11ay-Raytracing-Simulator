\appendix
\chapter*{Annexes}
\addcontentsline{toc}{chapter}{Annexes}
\renewcommand{\thesection}{\Alph{section}}

Ces annexes comprennent des parties du code du projet.
%\usemintedstyle{pastie}

\section{main.cpp}
\label{appendix:main.cpp}

Lance simplement l'interface utilisateur de l'aplication.

\inputminted[
frame=lines,
framesep=2mm,
baselinestretch=1.2,
fontsize=\footnotesize,
linenos,
breaklines
]{cpp}{../main.cpp}

% \begin{minted}[
% frame=lines,
% framesep=2mm,
% baselinestretch=1.2,
% fontsize=\footnotesize,
% linenos
% ]{python}

% for i in range(5):
%     print(i) # oui

% \end{minted}


\newpage
\section{simulation.cpp}
\label{appendix:simulation.cpp}
Ce fichier contient toute l'implémentation de la logique de la simulation.\\
Le \textit{header file \textbf{.h}} définissant la classe \textbf{Simulation} se trouve dans le \underline{\href{https://github.com/LucasPlacentino/802_11ay-Raytracing-Simulator}{Github}} du projet.
\inputminted[
frame=lines,
framesep=2mm,
baselinestretch=1.2,
fontsize=\footnotesize,
linenos,
breaklines
]{cpp}{../simulation.cpp}

\newpage
\section{transmitter.cpp}
\label{appendix:transmitter.cpp}
Ce fichier contient toute l'implémentation de la classe \textbf{Transmitter}.\\
Le \textit{header file \textbf{.h}} définissant la classe \textbf{Transmitter} se trouve dans le \underline{\href{https://github.com/LucasPlacentino/802_11ay-Raytracing-Simulator}{Github}} du projet.
\inputminted[
frame=lines,
framesep=2mm,
baselinestretch=1.2,
fontsize=\footnotesize,
linenos,
breaklines
]{cpp}{../transmitter.cpp}

\newpage
\section{receiver.cpp}
\label{appendix:receiver.cpp}
Ce fichier contient toute l'implémentation de la classe \textbf{Receiver}.\\
Le \textit{header file \textbf{.h}} définissant la classe \textbf{Receiver} se trouve dans le \underline{\href{https://github.com/LucasPlacentino/802_11ay-Raytracing-Simulator}{Github}} du projet.
\inputminted[
frame=lines,
framesep=2mm,
baselinestretch=1.2,
fontsize=\footnotesize,
linenos,
breaklines
]{cpp}{../receiver.cpp}

\newpage
\section{obstacle.cpp}
\label{appendix:obstacle.cpp}
Ce fichier contient toute l'implémentation de la classe \textbf{Obstacle} pour les murs.\\
Le \textit{header file \textbf{.h}} définissant la classe \textbf{Obstacle} se trouve dans le \underline{\href{https://github.com/LucasPlacentino/802_11ay-Raytracing-Simulator}{Github}} du projet.
\inputminted[
frame=lines,
framesep=2mm,
baselinestretch=1.2,
fontsize=\footnotesize,
linenos,
breaklines
]{cpp}{../obstacle.cpp}

\newpage
\section{ray.cpp}
\label{appendix:ray.cpp}
Ce fichier contient toute l'implémentation de la classe \textbf{Ray}.\\
Le \textit{header file \textbf{.h}} définissant la classe \textbf{Ray} se trouve dans le \underline{\href{https://github.com/LucasPlacentino/802_11ay-Raytracing-Simulator}{Github}} du projet.
\inputminted[
frame=lines,
framesep=2mm,
baselinestretch=1.2,
fontsize=\footnotesize,
linenos,
breaklines
]{cpp}{../ray.cpp}

\newpage
\section{raysegment.cpp}
\label{appendix:raysegment.cpp}
Ce fichier contient toute l'implémentation de la classe \textbf{RaySegment}.\\
Le \textit{header file \textbf{.h}} définissant la classe \textbf{RaySegment} se trouve dans le \underline{\href{https://github.com/LucasPlacentino/802_11ay-Raytracing-Simulator}{Github}} du projet.
\inputminted[
frame=lines,
framesep=2mm,
baselinestretch=1.2,
fontsize=\footnotesize,
linenos,
breaklines
]{cpp}{../raysegment.cpp}

\newpage
\section{parameters.h}
\label{appendix:parameters.h}
Ce fichier contient les paramètres de la simulation à définir lors de la compilation.
\inputminted[
frame=lines,
framesep=2mm,
baselinestretch=1.2,
fontsize=\footnotesize,
linenos,
breaklines
]{cpp}{../parameters.h}

% ---- pas nécessaire d'afficher mainwindow.cpp ? ----
%\newpage
%\section{mainwindow.cpp}
%\label{appendix:mainwindow.cpp}
%
%\inputminted[
%frame=lines,
%framesep=2mm,
%baselinestretch=1.2,
%fontsize=\footnotesize,
%linenos,
%breaklines
%]{cpp}{../mainwindow.cpp}

\newpage
\section{Autres parties du code}
\label{appendix:autres-codes}
D'autres parties du code sont disponibles sur le Github du projet:\\
\underline{\url{https://github.com/LucasPlacentino/802_11ay-Raytracing-Simulator}}.\\

Ceci inclut:
\begin{itemize}
    \item \textbf{mainwindow.cpp} (et son \textbf{.h}) : classe s'occupant de l'interface utilisateur.
    \item \textbf{tp4.cpp} : code reprenant l'implémentation de la simulation exclusivement utilisée pour le TP4, qui a servi de base initiale avant d'étendre les fonctionnalités au projet global.
    \item \textbf{CMakeList.txt} : le CMake généré et utilisé par \textit{QtCreator}.
\end{itemize}
