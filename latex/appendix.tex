\appendix
\chapter*{Annexes}
\addcontentsline{toc}{chapter}{Annexes}
\renewcommand{\thesection}{\Alph{section}}

Ces annexes comprennent des parties du code du projet.
%\usemintedstyle{pastie}

\section{main.cpp}
\label{appendix:main.cpp}

%TOUT LE CODE WESH

\inputminted[
frame=lines,
framesep=2mm,
baselinestretch=1.2,
fontsize=\footnotesize,
linenos
]{cpp}{../main.cpp}

% \begin{minted}[
% frame=lines,
% framesep=2mm,
% baselinestretch=1.2,
% fontsize=\footnotesize,
% linenos
% ]{python}

% for i in range(5):
%     print(i) # oui

% \end{minted}

\newpage
\section{mainwindow.cpp}
\label{appendix:mainwindow.cpp}

\inputminted[
frame=lines,
framesep=2mm,
baselinestretch=1.2,
fontsize=\footnotesize,
linenos
]{cpp}{../mainwindow.cpp}





\newpage
\section{Autres codes}
\label{appendix:autres-codes}
D'autres parties du code sont disponible sur le Github du projet:\\
\underline{\url{https://github.com/LucasPlacentino/802_11ay-Raytracing-Simulator}}.
