\clearpage\refstepcounter{chapter}%
\chapter*{Conclusion}
\addcontentsline{toc}{chapter}{Conclusion}
\label{conclusion}

En conclusion, ce projet de ray tracing pour le Wi-Fi IEEE 802.11ay a permis de développer un simulateur capable d'analyser la couverture réseau dans un environnement résidentiel prédéfini. Grâce à l'utilisation de la méthode des images et à l'utilisation du langage C++, le simulateur offre des résultats précis avec une erreur de moins de 0,5\% par rapport aux calculs manuels, et ce en un temps très réduit. La meilleure position pour la station de base, du point de vue pratique et théorique, a été identifiée, garantissant une couverture optimale de l'appartement. Bien que certaines limitations, comme l'absence de diffraction et les anomalies près de l'ascenseur, subsistent, le projet atteint ses objectifs principaux et propose une base solide pour des améliorations futures.
